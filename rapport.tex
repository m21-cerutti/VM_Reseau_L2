\documentclass[a4paper]{article}

\usepackage[T1]{fontenc}
\usepackage[utf8]{inputenc}
\usepackage{lmodern}

\usepackage[french]{babel}
\usepackage{graphicx}
\usepackage{xcolor}%				for colors
\usepackage{comment}%				for comment
\usepackage{upquote}%				for quote ''
\usepackage{hyperref}%				for HyperText
\hypersetup{
    colorlinks=true,
    linkcolor=black,
    filecolor=magenta,
    urlcolor=cyan,
}
\urlstyle{same}

\usepackage[toc,page]{appendix}

\usepackage{verbatim}%				Insertion de code
\usepackage{framed}%				Pour boite multipage
\usepackage{moreverb}%				Pour extension verbatim

\usepackage{listingsutf8}%				pour le code informatique
\definecolor{mGreen}{rgb}{0,0.6,0}
\definecolor{mGray}{rgb}{0.5,0.5,0.5}
\definecolor{mPurple}{rgb}{0.58,0,0.82}
\definecolor{backgroundColour}{rgb}{0.95,0.95,0.92}

\lstset{
	tabsize=4,
    extendedchars=true,
    breaklines=true,
    keepspaces=true,
	inputencoding=utf8,
    extendedchars=true,
    literate=%
                {é}{{\'e}}{1}%
                {è}{{\`e}}{1}%
                {à}{{\`a}}{1}%
                {ç}{{\c{c}}}{1}%
                {œ}{{\oe}}{1}%
                {ù}{{\`u}}{1}%
                {É}{{\'E}}{1}%
                {È}{{\`E}}{1}%
                {À}{{\`A}}{1}%
                {Ç}{{\c{C}}}{1}%
                {Œ}{{\OE}}{1}%
                {Ê}{{\^E}}{1}%
                {ê}{{\^e}}{1}%
                {î}{{\^i}}{1}%
                {ô}{{\^o}}{1}%
                {û}{{\^u}}{1}%
                {ë}{{\¨{e}}}1
                {û}{{\^{u}}}1
                {â}{{\^{a}}}1
                {Â}{{\^{A}}}1
                {Î}{{\^{I}}}1
}

\lstdefinestyle{PythonStyle}{
    backgroundcolor=\color{backgroundColour},
    commentstyle=\color{mGreen},
    keywordstyle=\color{magenta},
    numberstyle=\tiny\color{mGray},
    stringstyle=\color{mPurple},
    basicstyle=\footnotesize,
    breakatwhitespace=true,
    breaklines=true,
    captionpos=b,
    numbers=left,
    numbersep=5pt,
    language=Python
}

\title{Rapport pour Projet Réseau  TM1A}
\author{AMEEUW Vincent\and CERUTTI Marc}

%**************************************************************************

\begin{document}

\maketitle

\begin{abstract}
\center
\'Ebauche de rapport reseau \textit{Bomberman}\\
\end{abstract}

%Table

\tableofcontents

\newpage

%DOC
\part{Preambule}

\begin{comment}
Vous devez rendre ici deux fichiers :

    un rapport PDF (de 5 à 10 pages max, police 12) décrivant votre projet, organisé comme cela :
        introduction rapide au projet
        vue d'ensemble fonctionnelle : de ce qui marche parfaitement, de ce qui ne marche pas parfaitement, des bonus et de ce qui reste à faire, ... 
        architecture et implementation : vous prendrez soin de décrire les choix techniques que vous avez effectués (TCP/UDP, select/thread, recv non bloquant, acquittement, ...), ainsi que de décrire votre protocole réseau au moyen d'un formalisme de votre choix (shéma, algorithme en pseudo-langage, automate, ...). 
        bilan du projet
    une archive ZIP contenant tous les fichiers sources et ressources (images, ...) utiles au bon fonctionnement de votre projet, ainsi qu'un README détaillant comment lancer et jouer avec votre programme.

La date de rendu est fixée au \textbf{\textcolor{red}{vendredi 27 avril 23h55}}.


\end{comment}

\newpage
\part{Projet réseau}
	\section{Objectifs}
	
	\section{Méthode de travail}
	
	\section{Analyse du modèle}
	
	\section{Algorithme et implémentation}
		\subsection{Protocoles}
		
		\subsection{Choix techniques}
	
	
	\section{Améliorations effectues}
		\subsection{Collisions sur les bombes}
		\subsection{Gestion de déconnexion}
	
	\section{Bilan et critique}
		
\newpage
\appendix
\part{Annexes}


\section{Moodle} \label{moodle}

https://moodle1.u-bordeaux.fr/course/view.php?id=3671

\newpage
\section{Code Source}

\subsection{Network.py} \label{network.py}
\lstinputlisting[style=PythonStyle]{network.py}%firstline=1,lastline=2,


\end{document}
